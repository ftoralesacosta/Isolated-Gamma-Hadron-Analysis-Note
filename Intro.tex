\section{Motivation}
\label{sec:motivation}
The photon-tagged correlation of jets and jet fragments is a promising channel for the study of partonic energy loss in heavy-ion collisions. Energetic photons are free from the uncertainties that are associated with the fragmentation of partons into hadrons. However, existing measurements using high $E_{\mathrm{T}}$ photons focus on the study of energy loss beyond the region where the largest modification of particle spectra has been observed.

In this note, we present an analysis using pp and \pPb~data with the aim of bench-marking similar studies in Pb-Pb collisions. The comparison between \pPb~and Pb-Pb data disentangles effects due to the quark-gluon plasma (e.g. parton energy loss) and ``cold--nuclear matter'' effects such as modification of parton distribution functions in nuclei, and elastic, inelastic and coherent multiple parton scattering processes inside a large nucleus. This is because final-state effects associated with the quark-gluon plasma are expected to be absent or suppressed in \pPb~collisions.  

%While the quark nPDF of lead ions (Pb) is well understood from deep inelastic scattering data, the gluon PDFs, which is particularly important for perturbative QCD (pQCD) calculations at the CERN LHC energies, is not well constrained. Proton-nucleus collisions at LHC energies offers a unique access to the low-$x$ shadowing region in nuclei (below $x\approx 0.02$), which is severely unconstrained by data of deep-inelastic scattering and Drell-Yan production off nuclei.

%Measurements of hard-processes involving one heavy nucleus test factorization ``theorems", i.e they assign all nuclear effects into the initial conditions for the scale evolution of universal nPDFs. This assumption, which has had phenomenological success on existing data, is not proven nor expected to hold in general~\cite{deFlorian:2011fp}. The kinematic range and accuracy at which the nPDFs are known will continue to be a central issue in high energy nuclear physics, specially in the context of the opportunities offered by the future Electron Ion Collider~\cite{Accardi:2012qut}. 

%In this analysis we measure photons with $\pt$ in the range 15--30 \GeVc at mid-rapidity from collisions with $\sqrt{s_{\mathrm{NN}}}$ = 5 TeV to probe the region of $x_{\mathrm{T}} = 2\pt/\sqrt{s_{\mathrm{NN}}} \approx $ 0.006-0.012. This low-$x$ region can be probed with more energetic jets at large rapidities, as has been done by the ATLAS and CMS experiments at the LHC~\cite{Sirunyan:2018qel,Khachatryan:2015xaa,ATLAS:2014cpa}. However, the photon range probed in this analysis offers access to a much lower $Q^{2}$, which is closer to where the largest modifications are expected, and could further test the scale-evolution of nPDFs. 

%Compared with inclusive photon production, photon-jet correlations are more sensitive to parton density functions, photon and jet fragmentation, and their modifications in nuclei. Next-to-leading order calculations matched with parton showers (Pythia+POWHEG)~\cite{Klasen:2017dsy} that use different nuclear parton-density functions suggest that differences between the commonly used nPDFs result in large differences for the following observable that is sensitive to Bjorken-x in nuclei:
%\begin{equation}
%x_{\mathrm{obs}} = \frac{\left(\pt^{\gamma}e^{-\eta^{\gamma}} + \pt^{\mathrm{jet}}e^{-\eta^{\mathrm{jet}}}\right)}{2E_{\mathrm{Pb}}}. 
%\end{equation}
%The momentum and pseudorapidity ranges covered in this analysis, corresponds to $x_{\mathrm{obs}}$ in the range of 0.001--0.1. Similar variables have been used by the H1 and ZEUS experiments at the HERA electron-proton collider for determinations of proton and photon PDFs~\cite{Klasen:2002xb,Adloff:2000bs}.
