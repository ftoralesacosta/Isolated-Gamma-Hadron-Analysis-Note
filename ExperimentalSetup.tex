\section{Experimental Setup}
\label{sec:experimentalsetup}
A comprehensive description of the ALICE experiment and its performance is provided in Ref~\cite{Allen:2010stl,Abelev:2014ffa}. The detector elements most relevant for this study are the electromagnetic calorimeter system, which is used to measure and trigger on high $\pt$ photons, and the inner tracking system, which is used for tracking and vertexing. Both are located inside a large solenoid magnet with a field strength of 0.5 T. These are briefly described here:

The Electromagnetic Calorimeter (EMCal) is a sampling calorimeter composed of 77 alternating layers of {1.4 mm} lead and {1.7 mm} polystyrene scintillators. It has a cellular structure with square cells with a transverse size of 6 x 6 cm$^{2}$ called towers. The towers are arranged in a quasi-projective geometry. The tower transverse size is roughly equal to the Moli\`ere radius, so that most of the energy of the particle is deposited in one cell, about 90$\%$ for photons. It is located at 428 cm from the interaction point and its cell granularity is $\Delta\eta\times\Delta\varphi$ = 14.3$\times$14.3 mrad$^{2}$. It has an energy resolution is parametrized as $\sigma_{E}/E = 4.8\%/E\otimes 11.3\%/\sqrt{E}\otimes 1.7\%$ where the energy $E$ is given in units of GeV~\cite{Abeysekara:2010ze}. The linearity of the response of the detector and electronics has been measured with electron test beams to a precision better than 3$\%$ for the momentum range probed in this analysis. The non-linearity is negligible for cluster energy between 3 and 50 \GeVc, which is the relevant range for this analysis. The geometrical acceptance of the EMCal is $|\eta|<0.70$ and  $80^{\circ} < \varphi < 187^{\circ}$.

The Di-jet Calorimeter (DCal) is back-to-back in azimuth with respect to the EMCal. The DCal uses the same technology and material as the EMCal, thus having identical granularity and intrinsic energy resolution. It covers $0.22< \eta<0.7$, $260^{\circ}< \varphi <320^{\circ}$ and $|\eta|<$0.7, $320^{\circ}<\varphi<327^{\circ}$. It was installed and commissioned during the LHC long shutdown in 2015, and thus was operational during the 2017 pp run but not during the 2013 \pPb~run. %Data-driven studies, which are currently ongoing, with $\pi^{0}\to\gamma\gamma$ decays show that the absolute scale of DCal is within 2$\%$ of that of the EMCal.

The inner tracking system (ITS) consists of six layers of silicon detectors and is located directly around the interaction point. The two innermost layers consist of silicon pixel detectors positioned at radial distances of 3.9 cm and 7.6 cm, followed by two layers of silicon drift detectors at 15.0 cm and 23.9 cm, and two layers of silicon strip detectors at 38.0 cm and 43.0 cm. The ITS covers $|\eta|<0.9$ and has full azimuthal coverage. 

We also use the forward scintillators to provide the minimum-bias trigger and to estimate the particle multiplicity in each event. The V0 system consists of two scintillator arrays located on opposite sides of the interaction point at $z=-340$ cm and $z=+90$ cm covering $2.8 <\eta < 5.1$ and $-3.7 <\eta < -1.7$ respectively. It measures the total charge of the particles produced and the time of their arrival in each of the 64 channels.  