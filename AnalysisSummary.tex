\section{Analysis summary}
We use data collected during the \sqrtsNN{} = 5 TeV \pPb~run in 2013 and during the \sqrts{} = 5 TeV pp run in 2017. We use the EMCal trigger to select events with a high-momentum calorimeter cluster. For this analysis, we target photons with $\pt$ in the {12--40 \GeVc} range.%, equivalent to $x_{\mathrm{T}} = 2\pt/\sqrt{s_{\mathrm{NN}}}$ in the {0.006--0.012} range.  %The thresholds of the trigger corresponds to about {$\pt$ = 7 and 11 \GeVc} in the \pPb~data and about {5 \GeVc} in the pp data.

In this analysis, our signal are ``prompt" photons, which include ``direct photons" and ``fragmentation photons''. At leading order in perturbative QCD, the direct photons are produced in hard scattering processes such as quark-gluon Compton scattering ($qg\to q\gamma$) or quark-antiquark annihilation ($q\bar{q}\to g\gamma$), whereas the fragmentation photons are the product of the collinear fragmentation of a parton ($q\bar{q}(gg)\to \gamma + X$). At LHC energies, Compton scattering and gluon fusion $(gg\to  q\bar{q}\gamma)$ dominate due to the high-gluon density in the proton at small values of Bjorken-$x$. 

Beyond the simplistic leading order picture, the direct and fragmentation components have no physical meaning and cannot be factorized; the sum of their cross sections is the physical observable. For example, the separation between the NLO direct photons and LO fragmentation is arbitrary. However, it is still possible to simplify comparisons with theoretical calculations by applying an isolation criteria. We use an isolation variable that is the sum of the transverse momentum of the charged particles that are inside an angular cone of radius $R =\sqrt{(\Delta\phi)^{2} +(\Delta\eta)^{2}  } =0.4$ around the photon direction. 

The main background for our analysis are photons from meson decays, which we will call ``decay photons" or $\ydecay$. The  challenge that we face in this measurement arises mainly from the small cross-section of the signal compared to that of the decay photon background (about 1$\%$ at {10 \GeVc} increasing to about 4$\%$ at {30 \GeVc}, according to next-to-leading order calculations~\cite{Arleo:2004gn}). 

This measurement exploits the difference between the electromagnetic shower profiles of prompt photons and of photon pairs from neutral-meson decays. We call the clusters that pass our isolation and shower shape selections isolated $\gamma$ candidates or ``\gammaiso candidates". 

The main background in the $\gammaiso$ candidate sample arises
from multi-jet events where one jet typically contains a $\pi^{0}$ or $\eta$ that carries most of the jet energy and is misidentified as a photon because it decays into a photon pair that is collinear with respect to the EMCal cell granularity ($\Delta\eta\times\Delta\phi\approx$  14.3$\times$14.3 mrad$^{2}$), that is, the two photons are close enough to deposit most of their energy in the same cell. 

We measure the signal purity of our $\gammaiso$ selection by using the ``template-fit method", in which the measured shower-shape distribution is fit with the sum of signal and background templates with the relative normalization as the single free parameter\footnote{Note that this is an standard way to estimate QCD background since at least the Tevatron days. The same exact method is used in the CMS $\gammaiso$ and $\gammaiso$--jet measurements in pp and PbPb data, for example in Refs.~\cite{Sirunyan:2018gro,Chatrchyan:2012gt}.}. The background template is mostly data-driven, calculated with an anti-isolated sideband
requirement, but we apply a MC-based correction to account for estimated biases. The signal template is obtained from photon-jet simulation. The purity of our $\gammaiso$ selection is measured to be around 20$\%$ at {12 \GeVc} and increases to about 55$\%$ at {20 \GeVc} and above. 

We measure the angular correlation of our $\gammaiso$ candidates with charged particles. We correct for geometrical acceptance effects by using the mixed-event technique and then subtract the uncorrelated background, estimated by the zero-yield-at-minimum (ZYAM) method and by using a control region at large $|\eta^{\mathrm{hadron}}-\eta^{\gamma}|$. We measure the %correlated (BVJ)
{$\ydecay$--hadron} correlation function by inverting the shower-shape cut to select merged-clusters from meson decays. We normalize this correlation with the measured purity and subtract the normalized $\ydecay$--hadron correlation  background from the main $\gammaiso$ candidate correlations. Finally, we integrate the away-side of the resulting correlation function to determine the number of correlated hadrons per $\gammaiso$, i.e. to measure the conditional yield of hadrons. We perform this analysis with photons with $12<\pt <40~\GeVc$, and in intervals of charged particle \pt~and $\zt \equiv \pth/\ptgamma$. 

%We also present an $\gammaiso$--jet analysis. We reconstruct jets with the anti-$k_{\mathrm{T}}$ algorithm on tracks with {$0.15$ $<\pt < 15$ \GeVc} and $|\eta|<0.8$ as input. We select jets with {$\pt>$ 10 \GeVc} recoiling against the $\gammaiso$ candidate (with $|\phi^{\mathrm{jet}}-\phi^{\gamma}|>\pi/2$) and measure the angular correlations and momentum balance, and we also explore measurements sensitive to jet-flavor. To subtract the $\ydecay$--jet background arising from the impurity of our $\gammaiso$ candidate selection, we invert the shower-shape cut to select merged-clusters from meson decays and measure the corresponding background distributions, properly scaled by the purity of our $\gammaiso$ selection. We also subtract uncorrelated background, i.e correlations with jets produced by underlying event, by estimating it using event-mixing. The final $\gammaiso$--jet results are compared with $\pi^{0}$--jet correlations and \textsc{Pythia} photon+jet simulations at the reconstructed level and at particle level, i.e. after unfolding the jet detector smearing.

%The ratios of the p-Pb and pp distributions are compared to next-to-leading order perturbative quantum chromodynamic calculations with proton and nuclear parton distribution functions. While this measurement turns out to be dominated by statistical uncertainties, it still constrains the gluon densities in nuclei in the poorly explored low-$x$ and low-$Q^{2}$ region. This measurement also constitutes a benchmark for photon identification, background subtraction, and jet reconstruction for future measurements with larger data samples of proton-lead and lead-lead collisions. 

One of the novel aspects of this analysis is the use of ITS standalone tracking. We developed this approach to bypass the serious space-charge distortions that compromised the TPC during the high-luminosity \pPb~data taking in 2013\footnote{For details, see \url{https://alice.its.cern.ch/jira/browse/ATO-351}, \url{https://alice.its.cern.ch/jira/browse/PWGPP-349},
\url{https://alice.its.cern.ch/jira/browse/PWGPP-314}
}. Furthermore, the ITS-only tracking allowed the 2017 pp run to operate in the CALO mode that yielded a much larger sample than would have been possible otherwise. 

We have validated the performance of the ITS-standalone tracking (fake rate, efficiency and momentum smearing) by measuring the charged particle spectrum and comparing it with published ALICE measurements at the same center-of-mass energy. Our studies, included below, show agreement between the ITS-standalone measurement and the published data to within $\approx \pm 5\%$ of the corresponding published data for the range {$0.5<\pt<10$ \GeVc}, which is the relevant range in this analysis. 

One of the main considerations of our analysis strategy was to minimize the use of Monte Carlo simulations. By using an isolation variable constructed using only charged particles, we reduce the correlations between isolation and shower-shape variables due to the opening angle of neutral-meson decays, at the expense of a slightly lower purity. In our template fit analysis, we perform checks that are independent of any input from simulations, suggesting that we are not sensitive to the detailed simulation of the shape of the shower-shape distributions. Moreover, our analysis measures per-trigger quantities such that we do not need to correct for efficiency of the $\gammaiso$ selection. %While we use Monte Carlo simulations to construct the response matrices used in the unfolding of the detector response for jet measurements, these are validated with {\it in situ} calibrations of the jet energy scale. Therefore, detailed studies of multidimensional comparisons of data and simulations are not needed for this analysis, and are skipped in this note. 

While we made the effort to collect and use the largest data samples available by pioneering high-rate data taking with ITS+EMCal, our measurement turns out to be dominated by statistical uncertainties. Faced with this reality, we only make efforts to reduce the systematic uncertainties of the measurement such that they are smaller than the statistical uncertainty. 